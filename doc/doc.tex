\documentclass{article}

\usepackage{fullpage}
\usepackage{amsmath}
\usepackage{hyperref}

\author{Sam McSweeney}
\title{Miscellaneous Documentation for the Beamformer}

\begin{document}

\maketitle

\tableofcontents

\section{The ordering of the input files}

The heirarchy of the input data files is, nominally,
\[
    [\text{time sample}][\text{channel}][\text{antenna}][\text{polarisation}][\text{complexity}].
\]
For a second's worth of data in a single file, we have:
\[
    10000 \times 128 \times 128 \times 2 \times 2\quad (\times 4\,\text{bits}) = 327680000\,\text{bytes}
\]

However, the numbering of the antennas and the polarisations as the data come out of the PFB (and after the \texttt{recombine} stage) are different to the ordering assumed by the beamformer.
The $128 \times 2 = 256$ possible antenna/polarisation pairs are mapped as follows.
\begin{table}[!h]
    \centering
    \begin{tabular}{ccc|ccc}
        \multicolumn{3}{c}{Data file} & \multicolumn{3}{c}{Beamformer} \\
        ant & pol & index & index & ant & pol \\
        \hline
        0 & 0 & 0 &  0 &  0 & 0 \\
        0 & 1 & 1 & 16 &  8 & 0 \\
        1 & 0 & 2 & 32 & 16 & 0 \\
        1 & 1 & 3 & 48 & 24 & 0 \\
        2 & 0 & 4 &  1 &  0 & 1 \\
        2 & 1 & 5 & 17 &  8 & 1 \\
        3 & 0 & 6 & 33 & 16 & 1 \\
        3 & 1 & 7 & 49 & 24 & 1 \\
        4 & 0 & 8 &  2 &  1 & 0 \\
        \vdots & \vdots & \vdots & \vdots & \vdots & \vdots \\
        32 & 0 & 64 &  64 & 32 & 0 \\
        32 & 1 & 65 &  80 & 40 & 0 \\
        33 & 0 & 66 &  96 & 48 & 0 \\
        33 & 1 & 67 & 112 & 56 & 0 \\
        34 & 0 & 68 &  65 & 32 & 1 \\
        \vdots & \vdots & \vdots & \vdots & \vdots & \vdots \\
        64 & 0 & 128 & 128 & 64 & 0 \\
        64 & 1 & 129 & 144 & 72 & 0 \\
        65 & 0 & 130 & 160 & 80 & 0 \\
        65 & 1 & 131 & 176 & 88 & 0 \\
        66 & 0 & 132 & 129 & 64 & 1 \\
        \vdots & \vdots & \vdots & \vdots & \vdots & \vdots \\
    \end{tabular}
\end{table}

For efficiency, the beamformer should initially read in the whole second's worth of data in ``data order'', and calculate the index appropriate for a given antenna/polarisation pair as and when needed.
This requires finding the mapping from beamformer indices to data indices.
This becomes easier if we reconsider how we break up the $256$.

\newpage
\noindent If we recast the data file heirarchy\footnote{I'm not sure of the physical bases of this break up---I'm going off Steve Ord's's original code. rec=``Receiver'' and inc=``increment'' are my own terms.} as
\[
    [\text{time sample}][\text{channel}][\text{pfb}][\text{receiver}][\text{increment}][\text{complexity}],
\]
with the sizes being
\[
    10000 \times 128 \times 4 \times 16 \times 4 \times 2,
\]
then we can rewrite the table above as
\begin{table}[!h]
    \centering
    \begin{tabular}{cccc|ccc}
        \multicolumn{4}{c}{Data file} & \multicolumn{3}{c}{Beamformer} \\
        pfb & rec & inc & index & index & ant & pol \\
        \hline
        0 & 0 & 0 & 0 &  0 &  0 & 0 \\
        0 & 0 & 1 & 1 & 16 &  8 & 0 \\
        0 & 0 & 2 & 2 & 32 & 16 & 0 \\
        0 & 0 & 3 & 3 & 48 & 24 & 0 \\
        0 & 1 & 0 & 4 &  1 &  0 & 1 \\
        0 & 1 & 1 & 5 & 17 &  8 & 1 \\
        0 & 1 & 2 & 6 & 33 & 16 & 1 \\
        0 & 1 & 3 & 7 & 49 & 24 & 1 \\
        0 & 2 & 0 & 8 &  2 &  1 & 0 \\
        \vdots & \vdots & \vdots & \vdots & \vdots & \vdots & \vdots \\
        1 & 0 & 0 & 64 &  64 & 32 & 0 \\
        1 & 0 & 1 & 65 &  80 & 40 & 0 \\
        1 & 0 & 2 & 66 &  96 & 48 & 0 \\
        1 & 0 & 3 & 67 & 112 & 56 & 0 \\
        1 & 1 & 0 & 68 &  65 & 32 & 1 \\
        \vdots & \vdots & \vdots & \vdots & \vdots & \vdots & \vdots \\
        2 & 0 & 0 & 128 & 128 & 64 & 0 \\
        2 & 0 & 1 & 129 & 144 & 72 & 0 \\
        2 & 0 & 2 & 130 & 160 & 80 & 0 \\
        2 & 0 & 3 & 131 & 176 & 88 & 0 \\
        2 & 1 & 0 & 132 & 129 & 64 & 1 \\
        \vdots & \vdots & \vdots & \vdots & \vdots & \vdots & \vdots \\
    \end{tabular}
\end{table}

\newpage
\noindent Reversing this table (giving [ant, pol] in terms of [pfb, rec, inc]) yields
\begin{table}[!h]
    \centering
    \begin{tabular}{ccc|cccc}
        \multicolumn{3}{c}{Beamformer} & \multicolumn{4}{c}{Data file} \\
        index & ant & pol & pfb & rec & inc & index \\
        \hline
        0 &  0 & 0 & 0 & 0 & 0 & 0 \\
        1 &  0 & 1 & 0 & 1 & 0 & 4 \\
        2 &  1 & 0 & 0 & 2 & 0 & 8 \\
        3 &  1 & 1 & 0 & 3 & 0 & 12 \\
        4 &  2 & 0 & 0 & 4 & 0 & 16 \\
        \vdots & \vdots & \vdots & \vdots & \vdots & \vdots & \vdots \\
        16 &  8 & 0 & 0 & 0 & 1 & 1 \\
        17 &  8 & 1 & 0 & 1 & 1 & 5 \\
        18 &  9 & 0 & 0 & 2 & 1 & 9 \\
        19 &  9 & 1 & 0 & 3 & 1 & 13 \\
        20 & 10 & 0 & 0 & 4 & 1 & 17 \\
        \vdots & \vdots & \vdots & \vdots & \vdots & \vdots & \vdots \\
         64 & 32 & 0 & 1 & 0 & 0 & 64 \\
         65 & 32 & 1 & 1 & 1 & 0 & 68 \\
         66 & 33 & 0 & 1 & 2 & 0 & 72 \\
         67 & 33 & 1 & 1 & 3 & 0 & 76 \\
         68 & 34 & 0 & 1 & 4 & 0 & 80 \\
        \vdots & \vdots & \vdots & \vdots & \vdots & \vdots & \vdots \\
    \end{tabular}
\end{table}

\noindent From this we can see that the relationships between antenna, polarisation, pfb, receiver, and increment are
\begin{align*}
    \text{pfb} &= \text{ant} / 32 & & \text{(integer divide)} \\
    \text{rec} &= (2\times\text{ant}+\text{pol}) \% 16 & & \text{(modulus operator)} \\
    \text{inc} &= (\text{ant} / 8) \% 4
\end{align*}

\section{Beamformer operations}

By the time the data samples have been read in and are ready to be operated on, the beamformer expects two particular quantites to be already pre-calculated and available.
Both are listed in the table below, and both are calculated in \texttt{get\_delays()}.
\begin{table}[!h]
    \centering
    \begin{tabular}{c|l|l}
        Symbol & Variable name & Type \& Dimensions \\[5pt]
        \hline
        $W = \begin{bmatrix} w_X & 0 \\ 0 & w_Y \end{bmatrix}$ & \texttt{complex\_weights\_array} & \texttt{complex double [nstation*npol][nchan]} \\[12pt]
        $J^{-1} = \begin{bmatrix} j_{XX} & j_{XY} \\ j_{YX} & j_{YY} \end{bmatrix}$ & \texttt{invJi} & \texttt{complex double [nstation][npol*npol]}
    \end{tabular}
\end{table}
The individual components in the matrices $W$ and $J^{-1}$ are complex numbers\footnote{I have avoided using the inverse notation on the components of $J$ for brevity.}. The subscript indices denote polarisation, so that $w_X$, for example, can be thought of as only a function of antenna and channel number, while $j_{XX}$ is just a function of antenna.

A polarisation pair of samples, $D = \begin{bmatrix} d_X \\ d_Y \end{bmatrix}$ is then operated on as follows to produce the beam product
\[
    B = \sum_a J^{-1} W D,
\]
where the sum is over the antennas, giving a detected beam that is a function of channel number only.
Expressed in terms of the individual components with the dependence on antenna ($a$) and channel number ($c$) made explicit, we have
\[
    b_{P,c} = \sum_a \bigg(j_{PX,a} \, w_{X,a,c} \, d_{X,a,c} + j_{PY,a} \, w_{Y,a,c} \, d_{Y,a,c}\bigg)
\]

\end{document}
